\documentclass[modern]{aastex62}

% Load the corTeX style definitions
% !TeX root = ./ms.tex
% All the packages
\usepackage{url}
\usepackage{amsmath}
\usepackage{mathtools}
\usepackage{amssymb}
\usepackage{natbib}
\usepackage{graphicx}
\usepackage{calc}
\usepackage{etoolbox}
\usepackage{xspace}
\usepackage[T1]{fontenc} % https://tex.stackexchange.com/a/166791
\usepackage{textcomp}
\usepackage{ifxetex}
\ifxetex
  \usepackage{fontspec}
  \defaultfontfeatures{Extension = .otf}
\fi
\usepackage{fontawesome}
\usepackage{listings}
\usepackage{nicefrac}
%\usepackage{bm}
\usepackage{booktabs}
\usepackage{longtable}

% Shorthand for this paper
\newcommand{\starry}{\textsf{starry}\xspace}
\newcommand{\Python}{\textsf{Python}\xspace}
\newcommand{\xxx}[1]{{\color{red}#1}}
\newcommand{\quadquad}{\quad\quad\quad\quad}

% References to text content
\newcommand{\documentname}{\textsl{article}}
\newcommand{\figureref}[1]{\ref{fig:#1}}
\newcommand{\Figure}[1]{Figure~\figureref{#1}}
\newcommand{\figurelabel}[1]{\label{fig:#1}}
\renewcommand{\eqref}[1]{\ref{eq:#1}}
\newcommand{\Eq}[1]{Equation~(\eqref{#1})}
\newcommand{\eq}[1]{\Eq{#1}}
\newcommand{\eqalt}[1]{Equation~\eqref{#1}}

% Add code, proof, and animation hyperlinks
\definecolor{linkcolor}{rgb}{0.1216,0.4667,0.7059}
\definecolor{testpasscolor}{rgb}{0.13333333,0.5254902,0.22745098}
\definecolor{testmissingcolor}{rgb}{1.0,0.88,0.30}
\definecolor{testfailcolor}{rgb}{0.79607843,0.14117647,0.19215686}
\newcommand{\codeicon}{{\color{linkcolor}\faCloudDownload}}
\newcommand{\testmissingicon}{{\color{testmissingcolor}\faQuestion}}
\newcommand{\testpassicon}{{\color{testpasscolor}\faCheck}}
\newcommand{\testfailicon}{{\color{testfailcolor}\faTimes}}
\input{gitlinks}

% Define a proof environment for open source equation proofs
\newtagform{eqtag}[]{(}{)}
\newcommand{\currentlabel}{None}
\newenvironment{proof}[1]{%
  \ifstrempty{#1}{%
    \renewtagform{eqtag}[]{\raisebox{-0.1em}{{\testmissingicon}}\,(}{)}%
  }{%
    \renewtagform{eqtag}[]{\prooflink{#1}\,(}{)}%
  }%
  \usetagform{eqtag}%
  \renewcommand{\currentlabel}{#1}
  \align%
}{%
  \endalign%
  \renewtagform{eqtag}[]{(}{)}%
  \usetagform{eqtag}%
  \message{<<<\currentlabel: \theequation>>>}%
}

% Define the `oscaption` command for open source figure captions
\newcommand{\oscaption}[2]{\caption{#2 \codelink{#1}}}

% Code examples
\definecolor{codegreen}{rgb}{0,0.6,0}
\definecolor{codegray}{rgb}{0.5,0.5,0.5}
\definecolor{codepurple}{rgb}{0.58,0,0.82}
\definecolor{backcolour}{rgb}{0.95,0.95,0.95}
\lstdefinestyle{mystyle}{
  backgroundcolor=\color{backcolour},
  commentstyle=\color{codegreen},
  keywordstyle=\color{magenta},
  numberstyle=\tiny\color{codegray},
  stringstyle=\color{codepurple},
  basicstyle=\small\ttfamily,
  breakatwhitespace=false,
  breaklines=true,
  captionpos=b,
  keepspaces=true,
  numbers=left,
  numbersep=5pt,
  showspaces=false,
  showstringspaces=false,
  showtabs=false,
  tabsize=2,
  aboveskip=1em,
  belowskip=1em,
  keywords=[2]{map},
  keywordstyle=[2]{\color{black!80!black}},
  upquote=true
}
\lstset{style=mystyle}

% Typography obsessions
\setlength{\parindent}{3.0ex}
\renewcommand\quad{\hskip\fontdimen3\font}

% https://tex.stackexchange.com/a/184474
\usepackage{stackengine,scalerel}
\def\lnlam{\ThisStyle{\ensurestackMath{\stackon[-2.4\LMpt]{%
        \SavedStyle\lambda}{\kern-.5pt\kern\LMpt\rule{1\LMex}{.25pt+.15\LMpt}}}}}

% Load custom style
% !TeX root = ./ms.tex

% Packages
\usepackage{xifthen}
\usepackage{stackengine}
\usepackage{tabstackengine}
\usepackage{array}
\usepackage{upgreek}
\usepackage[bbgreekl]{mathbbol}
\usepackage{afterpage}
\usepackage[bb=boondox]{mathalpha}

% Misc. macros
\newcommand{\LMAX}{15\xspace}

% Integrals
\newcommand{\dd}{\ensuremath{\text{d}}}

% Special functions
\newcommand{\sgn}{{\text{sgn}}}
\newcommand{\atantwo}{{\text{arctan2}}}
\newcommand{\imag}{{\ensuremath{\mathbb{i}}}}

% Cartesian unit vectors
\newcommand{\xhat}{\ensuremath{\pmb{\hat{x}}}\xspace}
\newcommand{\yhat}{\ensuremath{\pmb{\hat{y}}}\xspace}
\newcommand{\zhat}{\ensuremath{\pmb{\hat{z}}}\xspace}

% Other
\DeclarePairedDelimiter\ceil{\lceil}{\rceil}
\DeclarePairedDelimiter\floor{\lfloor}{\rfloor}

% Inverse diagonal dots
\makeatletter
\def\Ddots{\mathinner{\mkern1mu\raise\p@
                \vbox{\kern7\p@\hbox{.}}\mkern2mu
                \raise4\p@\hbox{.}\mkern2mu\raise7\p@\hbox{.}\mkern1mu}}
\makeatother

% 
\DeclareSymbolFont{boldoperators}{OT1}{cmr}{bx}{n}
\SetSymbolFont{boldoperators}{bold}{OT1}{cmr}{bx}{n}
\edef\bbar{\unexpanded{\protect\mathaccentV{bar}}\number\symboldoperators16}
\edef\btilde{\unexpanded{\protect\mathaccentV{tilde}}\number\symboldoperators7E}


\usepackage{etoolbox}
\makeatletter % we need to patch \env@cases that has @ in its name
\patchcmd{\env@cases}{\quad}{\qquad\qquad}{}{}
\makeatother

\usepackage{enumitem}

% Begin!
\begin{document}

% Title
\title{%
    \textbf{
        Expectation of normalized variable
    }
}

\author{Rodrigo Luger}

\section{The problem}

Let $\mathbf{x} = \left( x_0 \,\, \cdots \,\, x_{K-1} \right)^\top$
be a $K$-dimensional multivariate normal random variable with mean $\pmb{\mu}$
and covariance $\pmb{\Sigma}$:
%
\begin{align}
    \mathbf{x} \sim \mathcal{N}\left( \pmb{\mu}, \pmb{\Sigma} \right)
    \quad.
\end{align}
%
For simplicity, let us assume the mean is constant, i.e.,
%
\begin{align}
    \pmb{\mu} = \mu \, \mathbf{j}_K
\end{align}
%
where $\mathbf{j}_K$ is the vector of $K$ ones.
%
Now suppose we cannot observe $\mathbf{x}$ directly, but instead we can
observe the \emph{normalized} version of $\mathbf{x}$, which we will call
$\mathbf{y}$:
%
\begin{align}
    \mathbf{y}
     & \equiv \frac{\mathbf{x}}{\bar{x}}
    \nonumber                                                      \\[0.5em]
     & = \frac{\mathbf{x}}{\frac{1}{K}\sum\limits_{k=0}^{K-1} x_k}
    \quad.
\end{align}
%
The variable $\mathbf{y}$ is no longer normally distributed, but as long as
the variance of $\bar{x}$ is small, we can approximate it as such. Let the mean
and covariance of $\mathbf{y}$ be $\mathbf{m}$ and $\mathbf{C}$, respectively:
%
\begin{align}
    \mathbf{y} \mathrel{\dot\sim} \mathcal{N}\left( \mathbf{m}, \mathbf{C} \right)
    \quad.
\end{align}
%
By definition, the mean is unity, i.e,
%
\begin{align}
    \mathbf{m} = \mathbf{j}_K
    \quad.
\end{align}
%
What is the expression for $\mathbf{C}$?

\section{The solution}

The covariance is given by
%
\begin{align}
    \mathbf{C}
     & =
    \mathrm{E}\big[ \mathbf{y} \mathbf{y}^\top \big]
    - \mathbf{m}\mathbf{m}^\top
    \nonumber \\[0.5em]
     & =
    \mathrm{E}\left[ \frac{\mathbf{x}\mathbf{x}^\top}{\bar{x}^2} \right]
    - \mathbf{J}_K
    \quad,
\end{align}
%
where $\mathbf{J}_K$ is the $K\times K$ matrix of ones.
To evaluate this, it is convenient to write
%
\begin{align}
    \mathbf{x} = \pmb{\mu} + \mathbf{L} \mathbf{u}
    \quad,
\end{align}
%
where $\mathbf{L}$ is the lower Cholesky decomposition of $\pmb{\Sigma}$,
i.e.,
%
\begin{align}
    \pmb{\Sigma} = \mathbf{L}\,\mathbf{L}^\top
\end{align}
%
and $\mathbf{u}$ is a standard multivariate normal random variable,
%
\begin{align}
    \mathbf{u} \sim \mathcal{N}\left( 0, \mathbf{I}_K \right)
    \quad,
\end{align}
%
where $\mathbf{I}_K$ is the
$K \times K$ identity matrix.
The expression for the covariance $\mathbf{C}$ may now be written
%
\begin{align}
    \mathbf{C}
     & =
    \mathrm{E}\left[
        \frac{
            (\pmb{\mu} + \mathbf{L} \mathbf{u})(\pmb{\mu} + \mathbf{L} \mathbf{u})^\top
        }{
            \bar{x}^2
        }
        \right]
    - \mathbf{J}_K
\end{align}
%
The mean $\bar{x}$ is
%
\begin{align}
    \bar{x} & = \sum\limits_{j=0}^{K-1}(\mu_j + L_{i,j}u_j)
    \nonumber                                               \\
            & =
    \mu + \frac{1}{K}\mathbf{j}^\top \mathbf{L} \, \mathbf{u}
    \quad.
\end{align}
%
Plugging this in and dividing the numerator and denominator in the expecation
by $\mu^2$, we obtain
%
\begin{align}
    \mathbf{C}
     & =
    \frac{1}{\mu^2}
    \mathrm{E}\left[
        \frac{
            (\pmb{\mu} + \mathbf{L} \mathbf{u})(\pmb{\mu} + \mathbf{L} \mathbf{u})^\top
        }{
            \left(1 + \frac{1}{\mu K}\mathbf{j}^\top \mathbf{L} \, \mathbf{u}\right)^2
        }
        \right]
    - \mathbf{J}_K
    \nonumber \\[0.5em]
     & =
    \frac{1}{\mu^2}
    \mathrm{E}\left[
        \frac{
            \mathbf{J}_K
            +
            \mathbf{j}_K\mathbf{u}^\top \mathbf{L}^\top
            +
            \mathbf{L} \, \mathbf{u} \, \mathbf{j}_K^\top
            +
            \mathbf{L} \, \mathbf{u} \, \mathbf{u}^\top \, \mathbf{L}^\top
        }{
            \left(1 + \frac{1}{\mu K}\mathbf{j}^\top \mathbf{L} \, \mathbf{u}\right)^2
        }
        \right]
    - \mathbf{J}_K
    \quad.
\end{align}
%
The denominator in the expression above makes direct evaluation of the expectation
intractable. As long as
$\big| \frac{1}{\mu K}\mathbf{j}^\top \mathbf{L} \, \mathbf{u} \big| < 1$, we
can Taylor expand the covariance as
%
\begin{align}
    \mathbf{C}
     & =
    \frac{1}{\mu^2}
    \mathrm{E}\left[
        \Bigg(
        \mathbf{J}_K
        +
        \mathbf{j}_K\mathbf{u}^\top \mathbf{L}^\top
        +
        \mathbf{L} \, \mathbf{u} \, \mathbf{j}_K^\top
        +
        \mathbf{L} \, \mathbf{u} \, \mathbf{u}^\top \, \mathbf{L}^\top
        \Bigg)
        \left(
        \sum\limits_{n=0}^\infty
        (-1)^n (n + 1) x^n
        \right)
        \right]
    - \mathbf{J}_K
    \quad,
\end{align}
%
where we define the scalar quantity
%
\begin{align}
    x \equiv \frac{1}{\mu K}\mathbf{j}_K^\top \mathbf{L} \, \mathbf{u}
    \quad.
\end{align}
%
Since the expectation is a linear operator, we can write this as
%
\begin{align}
    \mathbf{C}
     & =
    \frac{1}{\mu^2}
    \sum\limits_{n=0}^\infty
    (-1)^n (n + 1)
    \,
    \mathrm{E}\left[
        x^n
        \Bigg(
        \mathbf{J}_K
        +
        \mathbf{j}_K\mathbf{u}^\top \mathbf{L}^\top
        +
        \mathbf{L} \, \mathbf{u} \, \mathbf{j}_K^\top
        +
        \mathbf{L} \, \mathbf{u} \, \mathbf{u}^\top \, \mathbf{L}^\top
        \Bigg)
        \right]
    - \mathbf{J}_K
    \nonumber \\[0.5em]
     & =
    \frac{1}{\mu^2}
    \sum\limits_{n=0}^\infty
    \frac{(-1)^n (n + 1)}{(\mu K)^n}
    \,
    \left(
    \mathbf{P}_n
    +
    \mathbf{Q}_n
    +
    \mathbf{Q}_n^\top
    +
    \mathbf{R}_n
    \right)
    - \mathbf{J}_K
    \quad,
\end{align}
%
where
%
\begin{align}
    \mathbf{P}_n & \equiv \mathrm{E}\left[ \left(\mathbf{j}_K^\top \mathbf{L} \, \mathbf{u}\right)^n \mathbf{J}_K \right]
    \nonumber                                                                                                                                                               \\[0.5em]
    \mathbf{Q}_n & \equiv \mathrm{E}\left[ \left(\mathbf{j}_K^\top \mathbf{L} \, \mathbf{u}\right)^n \mathbf{L} \, \mathbf{u} \, \mathbf{j}_K^\top \right]
    \nonumber                                                                                                                                                               \\[0.5em]
    \mathbf{R}_n & \equiv \mathrm{E}\left[ \left(\mathbf{j}_K^\top \mathbf{L} \, \mathbf{u}\right)^n \mathbf{L} \, \mathbf{u} \, \mathbf{u}^\top \, \mathbf{L}^\top \right]
    \quad.
\end{align}
%
We compute these in the sections below, where we will make use of the expression for the
$n^\mathrm{th}$ moment of the standard normal distribution:
%
\begin{align}
    \mathrm{E}\left[ u^n \right]
     & =
    \begin{cases}
        0   & n \, \mathrm{odd}
        \\
        g_n & n \, \mathrm{even}
    \end{cases}
\end{align}
%
for scalar $u \sim \mathcal{N}(0, 1)$, where
%
\begin{align}
    g_n \equiv \dfrac{n!}{2^\frac{n}{2} \left(\frac{n}{2}\right)!}
    \quad.
\end{align}
%
For future reference, we can apply this relation to compute expectations
of products of the components of a random normal vector $\mathbf{u}$:
%
\begin{align}
    \mathrm{E}\left[u_i \right]                    & = 0
    \\[1em]
    \mathrm{E}\left[u_i u_j\right]                 & = g_2
    \\[1em]
    \mathrm{E}\left[u_i u_j u_k\right]             & = 0
    \\[1em]
    \mathrm{E}\left[u_i u_j u_k u_l\right]         & =
    \begin{cases}
        g_4   & i = j = k = l
        \\
        g_2^2 & i = j, k = l, i \ne k
        \\
        g_2^2 & i = k, j = l, i \ne j
        \\
        g_2^2 & i = l, j = k, i \ne j
        \\
        0     & \mathrm{otherwise}
    \end{cases}
    \\[1em]
    \mathrm{E}\left[u_i u_j u_k u_l u_m\right]     & = 0
    \\[1em]
    \mathrm{E}\left[u_i u_j u_k u_l u_m u_n\right] & =
    \xxx{...}
\end{align}

\subsection{The $\mathbf{P}$ expectation}

Let us begin by computing the first several terms in $\mathbf{P}$.
%
\begin{align}
    \mathbf{P}_0 & = \mathrm{E}\left[ \mathbf{J}_K \right]
    \nonumber                                              \\
                 & = \mathbf{J}_K
\end{align}
%
since $\mathbf{J}_K$ is constant. Next,
%
\begin{align}
    \mathbf{P}_1 & = \mathrm{E}\left[ \left(\mathbf{j}_K^\top \mathbf{L} \, \mathbf{u}\right) \mathbf{J}_K \right]
    \nonumber                                                                                                      \\
                 & = \left(\mathbf{j}_K^\top \mathbf{L} \, \mathrm{E}\left[ \mathbf{u} \right]\right) \mathbf{J}_K
    \nonumber                                                                                                      \\
                 & = \mathbf{0}
\end{align}
%
since $\mathrm{E}\left[ \mathbf{u} \right] = \mathbf{0}$. The next term is
%
\begin{align}
    \mathbf{P}_2 & = \mathrm{E}\left[ \left(\mathbf{j}_K^\top \mathbf{L} \, \mathbf{u}\right)^2 \mathbf{J}_K \right]
    \nonumber                                                                                                                                                         \\
                 & = \mathrm{E}\left[\left(\mathbf{j}_K^\top \mathbf{L} \,  \mathbf{u} \mathbf{u}^\top  \mathbf{L}^\top \, \mathbf{j}_K \right)  \mathbf{J}_K \right]
    \nonumber                                                                                                                                                         \\
                 & = \left(\mathbf{j}_K^\top \mathbf{L} \, \mathrm{E}\left[ \mathbf{u} \mathbf{u}^\top \right] \mathbf{L}^\top \, \mathbf{j}_K \right)  \mathbf{J}_K
    \nonumber                                                                                                                                                         \\
                 & = \left(\mathbf{j}_K^\top \mathbf{L} \mathbf{L}^\top \, \mathbf{j}_K \right)  \mathbf{J}_K
    \nonumber                                                                                                                                                         \\
                 & = \mathrm{sum}(\mathbf{\Sigma}) \mathbf{J}_K
\end{align}
%
where we made use of the fact that $\mathrm{E}\left[ \mathbf{u} \mathbf{u}^\top\right] = \mathbf{I}_K$.
%
Next,
%
\begin{align}
    \mathbf{P}_3 & = \mathrm{E}\left[ \left(\mathbf{j}_K^\top \mathbf{L} \, \mathbf{u}\right)^3 \mathbf{J}_K \right]
    \nonumber                                                                                                                                                                                                       \\
                 & = \mathrm{E}\left[\left(\mathbf{j}_K^\top \mathbf{L} \,  \mathbf{u} \mathbf{u}^\top  \mathbf{L}^\top \, \mathbf{j}_K \, \mathbf{j}_K^\top \mathbf{L} \,  \mathbf{u} \right) \mathbf{J}_K \right]
    \nonumber                                                                                                                                                                                                       \\
                 & = \left(\mathbf{j}_K^\top \mathbf{L} \,  \mathrm{E}\left[\mathbf{u} \mathbf{u}^\top  \mathbf{L}^\top \, \mathbf{J}_K \mathbf{L} \,  \mathbf{u}\right] \right) \mathbf{J}_K
    \nonumber                                                                                                                                                                                                       \\
                 & = \mathbf{0}
\end{align}
%
since we may write the component at index $i$ of the vector within the expectation
as
\begin{align}
    \mathrm{E}\left[\mathbf{u} \mathbf{u}^\top  \mathbf{L}^\top \, \mathbf{J}_K \mathbf{L} \,  \mathbf{u}\right]_{i} = \sum\limits_{k,l}(\mathbf{L}^\top \mathbf{J}_K \mathbf{L})_{k,l} u_i u_k u_l = 0
    \quad,
    \nonumber
\end{align}
%
which is zero since it is the product of an odd number of random normal variables.
Finally,
%
\begin{align}
    \mathbf{P}_4 & = \mathrm{E}\left[ \left(\mathbf{j}_K^\top \mathbf{L} \, \mathbf{u}\right)^4 \mathbf{J}_K \right]
    \nonumber                                                                                                                                                                                                                                                           \\
                 & = \mathrm{E}\left[\left(\mathbf{j}_K^\top \mathbf{L} \,  \mathbf{u} \mathbf{u}^\top  \mathbf{L}^\top \, \mathbf{j}_K \, \mathbf{j}_K^\top \mathbf{L} \,  \mathbf{u} \, \mathbf{u}^\top  \mathbf{L}^\top \, \mathbf{j}_K \right) \mathbf{J}_K \right]
    \nonumber                                                                                                                                                                                                                                                           \\
                 & = \left(\mathbf{j}_K^\top \mathbf{L} \, \mathrm{E}\left[ \mathbf{u} \mathbf{u}^\top  \mathbf{L}^\top \, \mathbf{J}_K \, \mathbf{L} \,  \mathbf{u} \, \mathbf{u}^\top \right] \mathbf{L}^\top \, \mathbf{j}_K \right) \mathbf{J}_K
    \nonumber                                                                                                                                                                                                                                                           \\
                 & = \left(3 \mathbf{j}_K^\top \mathbf{L} \, \mathbf{L}^\top \mathbf{J}_K \mathbf{L} \, \mathbf{L}^\top \, \mathbf{j}_K \right) \mathbf{J}_K
    \nonumber                                                                                                                                                                                                                                                           \\
                 & = 3 \, \mathrm{sum}(\mathbf{\Sigma})^2  \mathbf{J}_K
\end{align}
%
since
\begin{align}
    \mathrm{E}\left[\mathbf{u} \mathbf{u}^\top  \mathbf{L}^\top \, \mathbf{J}_K \mathbf{L} \,  \mathbf{u} \,  \mathbf{u}^\top\right]_{i,j}
     & =
    \sum\limits_{k,l}(\mathbf{L}^\top \mathbf{J}_K \mathbf{L})_{k,l} u_i u_j u_k u_l
    \nonumber \\
     & =
    3 \sum\limits_{k,l}(\mathbf{L}^\top \mathbf{J}_K \mathbf{L})_{k,l} \delta_{ijkl}
    \nonumber \\
     & =
    3 (\mathbf{L}^\top \mathbf{J}_K \mathbf{L})_{i,j}
    \nonumber \quad.
\end{align}
%
Based on this pattern, the general term in $\mathbf{P}$ is
%
\begin{align}
    \mathbf{P}_n & =
    \mathbf{J}_K
    \begin{cases}
        0                                                                                             & n \, \mathrm{odd}
        \\
        \dfrac{n! \, \mathrm{sum}(\pmb{\Sigma})^\frac{n}{2}}{2^\frac{n}{2} \left(\frac{n}{2}\right)!} & n \, \mathrm{even}
        \quad.
    \end{cases}
\end{align}

\subsection{The $\mathbf{Q}$ expectation}
%
The first several terms in $\mathbf{Q}$ are given by
%
\setlength{\abovedisplayskip}{1em}
\begin{align}
    \mathbf{Q}_0 & = \mathrm{E}\left[ \mathbf{L} \, \mathbf{u} \, \mathbf{j}_K^\top \right]
    \nonumber                                                                                                                                                                            \\
                 & = \mathbf{L} \, \mathrm{E}\left[  \mathbf{u} \right] \, \mathbf{j}_K^\top
    \nonumber                                                                                                                                                                            \\
                 & = \mathbf{0}
    %
    \\[1em]
    %
    \mathbf{Q}_1 & = \mathrm{E}\left[ \left(\mathbf{j}_K^\top \mathbf{L} \, \mathbf{u}\right) \mathbf{L} \, \mathbf{u} \, \mathbf{j}_K^\top \right]
    \nonumber                                                                                                                                                                            \\
                 & = \mathrm{E}\left[ \left(\mathbf{j}_K^\top \mathbf{L} \, \mathbf{u}\right) \mathbf{L} \, \mathbf{u} \right] \, \mathbf{j}_K^\top
    \nonumber                                                                                                                                                                            \\
                 & = \mathrm{E}\left[ \mathbf{L} \, \mathbf{u} (\mathbf{u}^\top \mathbf{L}^\top \mathbf{j}_K) \right] \, \mathbf{j}_K^\top
    \nonumber                                                                                                                                                                            \\
                 & = \mathrm{E}\left[ \mathbf{L} \, \mathbf{u} \mathbf{u}^\top \mathbf{L}^\top  \right] \, \mathbf{J}_K
    \nonumber                                                                                                                                                                            \\
                 & = \mathbf{L} \, \mathrm{E}\left[  \mathbf{u} \mathbf{u}^\top   \right]  \, \mathbf{L}^\top \mathbf{J}_K
    \nonumber                                                                                                                                                                            \\
                 & = \mathbf{L} \, \mathbf{L}^\top \mathbf{J}_K
    \nonumber                                                                                                                                                                            \\
                 & = \pmb{\Sigma} \, \mathbf{J}_K
    %
    \\
    \intertext{%
        where we made liberal use of the fact that $\left(\mathbf{j}_K^\top \mathbf{L} \, \mathbf{u}\right)$ is a scalar,%
    }
    %
    \mathbf{Q}_2 & = \mathrm{E}\left[ \left(\mathbf{j}_K^\top \mathbf{L} \, \mathbf{u}\right)^2 \mathbf{L} \, \mathbf{u} \, \mathbf{j}_K^\top \right]
    \nonumber                                                                                                                                                                            \\
                 & =  \mathrm{E}\left[ \mathbf{L} \, \mathbf{u} \, (\mathbf{j}_K^\top \mathbf{L} \, \mathbf{u}) (\mathbf{u}^\top \mathbf{L}^\top \mathbf{j}_K) \mathbf{j}_K^\top \right]
    \nonumber                                                                                                                                                                            \\
                 & = \mathbf{L} \, \mathrm{E}\left[ \mathbf{u} \, \mathbf{j}_K^\top \mathbf{L} \, \mathbf{u} \mathbf{u}^\top \right] \, \mathbf{L}^\top \mathbf{J}_K
    \nonumber                                                                                                                                                                            \\
                 & = \mathbf{0}
    %
    \\
    \intertext{since $\mathrm{E}\left[ \mathbf{u} \, \mathbf{j}_K^\top \mathbf{L} \, \mathbf{u} \mathbf{u}^\top \right]_{i,j} = 0$,}
    %
    \mathbf{Q}_3 & = \mathrm{E}\left[ \left(\mathbf{j}_K^\top \mathbf{L} \, \mathbf{u}\right)^3 \mathbf{L} \, \mathbf{u} \, \mathbf{j}_K^\top \right]
    \nonumber                                                                                                                                                                            \\
                 & = \xxx{...}
    \nonumber                                                                                                                                                                            \\
                 & = 3 \, \pmb{\Sigma} \, \mathbf{J}_K \, \pmb{\Sigma} \, \mathbf{J}_K
\end{align}
%
Based on this pattern, the general term in $\mathbf{Q}$ is
%
\begin{align}
    \mathbf{Q}_n & =
    \begin{cases}
        \mathbf{0}                                                                                                          & n \, \mathrm{even}
        \\
        \dfrac{(n + 1)!}{2^\frac{n + 1}{2} \left(\frac{n + 1}{2}\right)!} \, (\pmb{\Sigma} \, \mathbf{J}_K)^\frac{n + 1}{2} & n \, \mathrm{odd}
        \quad.
    \end{cases}
\end{align}

\subsection{The $\mathbf{R}$ expectation}
%
The first several terms in $\mathbf{R}$ are given by
%
\setlength{\abovedisplayskip}{1em}
\begin{align}
    \mathbf{R}_0 & = \mathrm{E}\left[ \mathbf{L} \, \mathbf{u} \, \mathbf{u}^\top \mathbf{L}^\top \right]
    \nonumber                                                                                                                                                       \\
                 & = \mathbf{L} \, \mathrm{E}\left[  \mathbf{u} \, \mathbf{u}^\top \right] \, \mathbf{L}^\top
    \nonumber                                                                                                                                                       \\
                 & = \pmb{\Sigma}
    %
    \\[1em]
    %
    \mathbf{R}_1 & = \mathrm{E}\left[ \left(\mathbf{j}_K^\top \mathbf{L} \, \mathbf{u}\right) \mathbf{L} \, \mathbf{u} \, \mathbf{u}^\top \mathbf{L}^\top \right]
    \nonumber                                                                                                                                                       \\
                 & = \xxx{...}
    \nonumber                                                                                                                                                       \\
                 & = \mathbf{0}
    %
    \\[1em]
    %
    \mathbf{R}_2 & = \mathrm{E}\left[ \left(\mathbf{j}_K^\top \mathbf{L} \, \mathbf{u}\right)^2 \mathbf{L} \, \mathbf{u} \, \mathbf{u}^\top \mathbf{L}^\top \right]
    \nonumber                                                                                                                                                       \\
                 & = \xxx{...}
    \nonumber                                                                                                                                                       \\
                 & = 3 \, \pmb{\Sigma} \, \mathbf{J}_K \, \pmb{\Sigma}
\end{align}
\end{document}